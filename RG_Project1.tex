\documentclass{article}
\usepackage[utf8]{inputenc}
\usepackage{amsmath}
\usepackage{geometry}
\geometry{margin=1in}

\title{\vspace{-5.5cm}Random Graphs Project 1}
\author{Alex Rose, Pedro}
\date{February 2016}

\begin{document}

\maketitle

\section{Warm-Up Identities}

\subsection{}
$$ C'(z) =\sum_{n \ge 0} \frac{c_n}{n!} * \frac{d}{dz} z^n = \sum_{n \ge 1} \frac{c_n}{n!}nz^{n-1}$$ 
$$\newline = \sum_{n \ge 1} \frac{c_n}{(n-1)!}z^{n-1} = \sum_{n \ge 0} \frac{c_{n+1}}{n!}z^{n} = (c_{n+1})_{n \ge 0} $$

\subsection{}
$$ \int_{0}^{z} C(t) dt  =\sum_{n \ge 0} \frac{c_n}{n!} * \int_{0}^{z} t^n dt = \sum_{n \ge 1} \frac{c_n}{n!} \frac{z^{n+1}}{n+1}$$ 
$$\newline = \sum_{n \ge 0} \frac{c_n}{(n+1)!}z^{n+11} = \sum_{n \ge 1} \frac{c_{n-1}}{n!}z^{n} = (c_{n-1})_{n \ge 1} $$ 

\section{Pointed Class}
For each object of size n, we have n ways of choosing the pointer, one for each element in the object. So the pointed class generating function is of the form:

$$C^\swarrow(z) = \sum_{n \ge 0} n\frac{c_n}{n!}z^n = \sum_{n \ge 0} \frac{c_n}{(n-1)!}zz^{n-1} = z\sum_{n \ge 0} \frac{c_n}{(n-1)!}z^{n-1} = z\sum_{n \ge 0} \frac{c_{n+1}}{n!}z^{n}$$

And from (1.1), the remaining sum is equal to $C'(z)$. So $C^\swarrow(z) = zC'(z)$.


\section{Modified Labelled Product}

\subsection{}
The recursion for the \textit{unlabelled} $c_n$, in terms of $a_n$ and $b_n$, is given by:

$$c_n = \sum_{i + k = n} a_ib_k$$

As a particular index must be in A, we choose $k$ labels from a pool of $n-1 = i + k - 1$ labels for the B-member. Thus our labelled recursion is of the form:

$$c_n = \sum_{i + k = n} \binom{i + k - 1}{k}a_ib_k$$

\subsection{}

Note that:

$$C(z) =  \int_{0}^{z} C'(t) dt = \int_{0}^{z} \sum_{n \ge 0} \frac{c_{n+1}}{n!}t^{n}  = \int_{0}^{z} \sum_{n \ge 1} \frac{c_n}{(n-1)!}t^{n-1}  dt$$ 

We seek the value of the inner sum, $C'(z)$. Substituting $c_n$ for the recurrence found in 3.1:

$$\sum_{n \ge 1}\frac{c_n}{(n-1)!}t^{n-1}  = \sum_{n \ge 1}\frac{1}{(n-1)!} \bigg( \sum_{i + k = n} \binom{i + k - 1}{k}a_ib_k\bigg)t^{n-1}$$

We can separate the variables in the binomial coefficient.

$$\binom{i + k - 1}{k} = \frac{(i+k-1)!}{k!(i + k - 1 - k)!} = \frac{(n-1)!}{k!(i-1)!} $$

We then move the $(n-1)!$ term to the outer sum, cancelling the $\frac{1}{(n-1)!}$ term.

$$\sum_{n \ge 1}\frac{c_n}{(n-1)!}t^{n-1}  = \sum_{n \ge 1} \bigg( \sum_{i + k = n} \frac{a_ib_k}{(i-1)!k!}\bigg)t^{n-1}$$

We now separate the variables in this inner sum, then (appropriately) distribute the t power into the inner sum:

$$ = \sum_{n \ge 1} \bigg( \sum_{i + k = n} \frac{a_it^{i-1}}{(i-1)!} * \frac{b_kt^k}{k!}\bigg)$$

Where each term represents the convolution of $(a_{i+1})_{i \ge 0}$ and $(b_{k})_{k \ge 0}$. Thus $C'(t) = A'(t) \cdot B(t)$, and so:

$$C(z) = \int_{0}^{z} A'(t) \cdot B(t) dt$$

\section{Records}

\subsection{}
Consider $(\{1\}^+\otimes \mathcal{P})$. This means all the permutations of $\mathcal{P}$ where the element $\{1\}$ has the highest label or in other words it is the start of a record. The power to the $k$ is the cross product of $k$ of these combinatorial classes. This order matters in this formulation, but the only valid ordering of the records is lowest to highest. Thus our generated set is effectively unordered for our purposes, which is captured by the transformation to a multiset. 

\subsection{} 

From 3.2:

$$(\{1\}^+\otimes \mathcal{P})^k(z) = \left(\int_{0}^{z} \{1\}'(t) \cdot P(t) dt\right)^k = \left(\int_{0}^{z} 1 \cdot \frac{1}{1-z}  dt\right)^k = \log(1-z)^k $$ 

As discussed in the previous section, the "input" order forced by the cross product is not relevant for our purposes. Taking the multiset allows us to disregard this initial ordering, but then overcounts by exactly $\frac{1}{k!}$ since there is exactly one valid ordering. Therefore the result is:

As discussed in the previous section, the "input" order forced by the cross product is not relevant for our purposes. So this ordering of k record segments overcounts by $k!$. The order-ignoring multiset corrects for this and scales down the above expression by $\frac{1}{k!}$, so $MSet(\{1\}^+\otimes \mathcal{P})^k(z) = \frac{1}{k!}\log(1-z)^k$.

\subsection{}

Note $P_k(z)$ counts the number of n-sized elements with k records, with k fixed over the univariate generating function: 

$$P_k(z) =  \frac{c_1}{1!}z^1 + \frac{c_2}{2!}z^2 +... = \frac{r_{1,k}}{1!}z^1 + \frac{r_{2,k}}{2!}z^2 +... = \sum_{n\ge 0} \frac{r_{n,k}}{n!}z^n$$

We can swap the summations of $\mathcal{P}(z,x)$:
$$\mathcal{P}(z,x) =\sum_{k\ge 1}\left( \sum_{n\ge 0} \frac{r_{n,k}}{n!}z^n\right) x^k = \sum_{k\ge 1} P_k(z) x^k $$

Substituting the generating function found in 4.2 yields:

\begin{align*}
	\mathcal{P}(z,x)&=\sum_{n\ge 0, k\ge 1}\frac{r_{n,k}}{n!}z^nx^k\\
	&=\sum_{k\ge 1}\mathcal{P}_k(z) x^k\\
	&=\sum_{k\ge 1}\frac{1}{k!}\left(\log\frac{1}{1-z}\right)^kx^k=\sum_{k\ge 1}\frac{1}{k!}\left(x\log\frac{1}{1-z}\right)^k\\
	&=e^{xlog(\frac{1}{1-z})} = e^{log(\frac{1}{(1-z)^x})} = (1-z)^{-x}
\end{align*}

\subsection{}
Take the $m$-th derivative of $\mathcal P(x,z)$ with respect to z. Note this eliminates the first $m$ terms.

$$\frac{\partial^m \mathcal P(x,z)}{\partial z^m} = \sum_{n \ge m} \sum_{k = 1}^{n} \frac{r_{n,k}}{n!}\frac{n!}{(n-m)!}x^kz^{n-m}$$. 

In particular, the $m$-th term is $\sum_{k = 1}^{m}r_{m,k} x^k$ for all $z$. If $z = 0$, then this term is the only nonzero term remaining. Thus:

$$\sum_{k = 1}^{n}r_{m,k} x^k = \frac{\partial^m \mathcal P}{\partial z^m}(0,x) $$

Applying these operations to the right hand side, $\mathcal P(x,z) = (1-z)^{-x}$:

$$\frac{\partial}{\partial z^m} (1-z)^{-x}|_{z = 0} = (-1)^m(-x)(-x-1)(-x-2)....(-x - m + 1)(1-z)^{-x-m}|_{z = 0}$$
$$= (-1)^m(-1)^m(x)(x+1)(x+2)...(x + m -1)(1-z)^{-x-m}|_{z = 0}$$
$$= (x)(x+1)(x+2)...(x + m -1)$$

Which is our the result.

\subsection{}

Define the above expression as $F(x)$. From principles, the expected number of records in an $n$ size permutation is given by $F'(1)/F(1)$. $F(1)$ is straightforward to calculate:

$$F(1) = 1(1+1)(1 + 2)....(1 + m - 1) = m!$$

To find F'(1), Define a new function $G_j$ as:

$$G_j(x) = (x+m-1)(x+m-2)....(x+m-j), G_0(x) = 0$$

In particular:

$$G_m(x) = (x+m-1)(x+m-2)...(x+m-j) = F(x)$$ 
$$G_j(1) = m(m-1)...( (m-j)+1) = \frac{m!}{(m-j)!}$$

We can express the derivative of $G_m(x)$ as two parts via the power rule:

$$G_k(x)|_{x = 1} = (x+m-k)' G_{k-1}(x) + (x+m-k)G'_{k-1}(x)|_{x = 1}$$
$$= G_{k-1}(1) + (m+1-k)*G'_{k-1}(x)|_{x = 1}$$

So we have a recurrence, with $R(k) = G_{k}(x)|_{x = 1}$:

$$R(k) = \frac{m!}{(m+1-k)!} + (m+1-k)R(k-1),   R(0) = 0$$

Evaluating the recursion at $m$ eventually gives:

$$F'(1) = R(m) = \sum_{k = 1}^{m} \frac{m!}{(m+1-k)!}*(m+1-(k+1))(m+1-(k+2))...(2)(1)$$
$$= \sum_{k = 1}^{m} \frac{m!}{(m-k+1)!}(m-k)! = m!\sum_{k = 1}^{m}\frac{1}{m+1-k} = m!\sum_{j = 1}^{m}\frac{1}{j}$$

And so the expected number of runs is:

$$E = F'(1)/F(1) = \frac{m!\sum_{j = 1}^{m}\frac{1}{j}}{m!} = \sum_{j = 1}^{m}\frac{1}{j} $$

I realized about halfway through this was complete overkill. 


\end{document}
